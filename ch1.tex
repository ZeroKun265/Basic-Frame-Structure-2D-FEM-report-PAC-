\begin{figure}[H]
    \centering

    \begin{minipage}{0.3\linewidth}
        \centering
        \includegraphics[width=\linewidth]{Images/carr.png}
        \caption{Carrello a terra}
        \label{fig:carr}
    \end{minipage}
    \hfill
    \begin{minipage}{0.3\linewidth}
        \centering
        \includegraphics[width=\linewidth]{Images/Banchetto prova.png}
        \caption{Banco di prova con sistema di carico}
        \label{fig:banco}
    \end{minipage}
    \hfill
    \begin{minipage}{0.3\linewidth}
        \centering
        \includegraphics[width=\linewidth]{Images/cern.png}
        \caption{Cerniera a terra}
        \label{fig:cern}
    \end{minipage}

\end{figure}

Nel seguente rapporto, vogliamo studiare le deformazioni di una struttura a travi di un banco di prova, rappresentata nella figura \ref{fig:banco}.
Modelliamo le aste con elementi truss lineari (T3D2) e elementi beam lineari (B21) utilizzando il risolutore di simulazioni Abaqus e adottando un approccio analitico (solo per i truss).

Le 7 aste sono collegate con delle piastre bullonate, di cui le due di sinistra sono vincolate a terra con una cerniera (per quella inferiore) e un carrello (per quella superiore). Un carico di $428 \space N$ viene applicato dal sistema di carico a destra.

\section{Ottenimento dei vincoli}
\label{ss:ottenimento_vincoli}%
Le aste sono vincolate con delle piastre di giunzione bullonate, come in figura \ref{fig:giunz}.
Queste possono agire da vincolo di incastro quando ai bulloni viene applicata una coppia di serraggio di $10 \space Nm$, oppure da vincolo di cerniera se non sono serrati.

Otteniamo i vincoli di carrello e cerniera a terra con due montaggi a terra diversi, rispettivamente mostrati in figura \ref{fig:carr} e figura \ref{fig:cern}


\section{Travi e giunzioni}
\label{ss:travi_giunzioni}%
\begin{figure}[H]
    \centering

    \begin{minipage}{0.40\linewidth}
        \centering
        \includegraphics[width=0.6\linewidth]{Images/giunzione.png}
        \caption{Piastra di giunzione in attivo}
        \label{fig:giunz}
    \end{minipage}
    \hfill
    \begin{minipage}{0.40\linewidth}
        \centering
        \includegraphics[width=0.6\linewidth]{Images/Piastra tavola.png}
        \caption{Quote della piastra di giunzione}
        \label{fig:giunz_tavola}
    \end{minipage}

\end{figure}

Le varie dimensioni della piastra sono rappresentate nella figura \ref{fig:giunz_tavola}. I bulloni usati sono M6.
Nella figura \ref{fig:asta_giunta} abbiamo la rappresentazione del sistema di fissaggio delle aste.
\begin{figure}[H]
    \centering
    \includegraphics[width=0.50\linewidth]{Images/sist fissaggio.png}
    \caption{Sistema di fissaggio delle aste}
    \label{fig:asta_giunta}
\end{figure}

Le travi sono in alluminio, con il modulo di Young $E = 61639.8 MPa$, sono di sezione quadrata cava, con le quote riportate in figura \ref{fig:aste_quote}


\begin{figure}[H]
    \centering
    \includegraphics[width=1\linewidth]{Images/asta_quote.png}
    \caption{Quote di asta e sezione trasversale}
    \label{fig:aste_quote}
\end{figure}

\section{Posizionamento e utilizzo degli estensimetri}
\label{s:estensimetri}%
Posizioniamo degli estensimetri sulle aste della struttura, collegati a diversi canali, numerati da 0 a 7, connessi poi a una scheda di acquisizione \textit{National Instrument NI-9235} con software \textit{LabVIEW}.
Ogni estensimetro è di tipo \textit{1-LY13-6/120} , specifico per leghe di alluminio, montati come nell'immagine \ref{fig:est_montaggio}. Il circuito di misura è rappresentato nella figura \ref{fig:schema_wheatstone}

\begin{figure}[H]
    \centering
    \begin{tikzpicture}[scale=0.9]
\draw (-3,-2) to[battery] (-3, 2);

\draw (-3, 2) -- (0, 2); %to top
\draw (-3, -2) -- (0, -2);%to bot

\draw (0, 2) -- (0, 1.5);
\draw (0, -2) -- (0, -1.5);

\draw (0, 1.5) to[R=$R_1$] (-2, 0);
\draw (0, -1.5) to[R=$R_3$] (-2, 0);

\draw (0, 1.5) to[R=$R_2$] (2, 0);

\draw (0, -1.5) -- (3,-1.5);
\draw (0,0) node[draw, circle, minimum size=0.5cm] (v) {$V$};
\draw (-2, 0) -- (-0.4,0);
\draw (0.4, 0) -- (1,0);
\draw (1, 0) -- (1, -0.8);

\draw (2,0) -- (3,0);
\draw (3, 0) to[R=$R_{w,1}$] (4,0);
\draw (1, -0.8) -- (3, -0.8);
\draw (3, -0.8) to[R=$R_{w,2}$] (4,-0.8);

\draw (4, -0.8) -- (4.2, -0.8);
\draw (4, 0) -- (4.2, 0);
\draw (4.2, 0) -- (4.2, -0.8);

\draw (3, -1.5) to[R=$R_{w,3}$] (4, -1.5);
\draw (4, -1.5) -- (5.2, -1.5);

\draw (4.2, -0.4) -- (5.2, -0.4);

\draw (5.2, -0.4) -- (5.4, -0.4);
\draw (5.2, -1.5) -- (5.4, -1.5);

\fill (5.2,-0.4) circle (3pt);
\fill (5.2,-1.5) circle (3pt);

\draw (5.4, -0.4) -- (7, -0.4);
\draw[ultra thick] (7, -0.4) -- (7, -0.6);
\draw (7, -0.6) -- (5.4, -0.6);
\draw[ultra thick] (5.4, -0.6) -- (5.4, -0.8);
\draw (5.4, -0.8) -- (7, -0.8);
\draw[ultra thick] (7, -0.8) -- (7, -1);
\draw (7, -1) -- (5.4, -1);
\draw[ultra thick] (5.4, -1) -- (5.4, -1.3);
\draw (5.4, -1.3) -- (7, -1.3);
\draw[ultra thick] (7, -1.3) -- (7, -1.5);
\draw (7, -1.5) -- (5.4, -1.5);

%boxes
\draw[color=red] (-3.5, 3) rectangle (2.7, 0-3);
\draw [ color=red](-0.75,3.3) to[short] (-0.75,3.3) node {Scheda acquisizione};

\draw[color=blue] (2.9, 1.2) rectangle (4.6, -2.5);
\draw [ color=blue](3.75, 1.4) to[short] (3.75, 1.4) node {Fili};

\draw[color=purple] (4.9, 0.1) rectangle (7.3, -2);
\draw [color=purple](6.1, 0.3) to[short] (6.1, 0.4) node {Estensimetro};

\end{tikzpicture}
\caption{Schema del ponte di Wheatstone con estensimetro a 3 fili}
\label{fig:schema_wheatstone}
\end{figure}

I nostri estensimetri sono posizionati, uno per asta, come nell'immagine \ref{fig:posiz_est}


\begin{figure}[H]
    \centering

    \begin{minipage}{0.45\linewidth}
        \centering
        \includegraphics[width=\linewidth]{Images/Montaggio estensimetro.png}
        \caption{Montaggio estensimetro}
        \label{fig:est_montaggio}
    \end{minipage}
    \hfill
    \begin{minipage}{0.45\linewidth}
        \centering
        \includegraphics[width=0.65\linewidth]{Images/posizioni_est.png}
        \caption{Posizioni degli estensimetri}
        \label{fig:posiz_est}
    \end{minipage}

\end{figure}

I dettagli della configurazione corrente sono:
\begin{itemize}
    \item Gli estensimetri misurano uno alla volta
    \item Gli effetti termici sull'estensimetro non sono compensati
    \item Le resistenze dei fili non sono compensate
\end{itemize}