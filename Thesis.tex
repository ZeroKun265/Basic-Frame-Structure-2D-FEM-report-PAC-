\documentclass{Configuration_Files/PoliMi3i_thesis}

%------------------------------------------------------------------------------
%	REQUIRED PACKAGES AND  CONFIGURATIONS
%------------------------------------------------------------------------------

% CONFIGURATIONS
\usepackage{parskip} % For paragraph layout
\usepackage{setspace} % For using single or double spacing
\usepackage{emptypage} % To insert empty pages
\usepackage{multicol} % To write in multiple columns (executive summary)
\setlength\columnsep{15pt} % Column separation in executive summary
\setlength\parindent{0pt} % Indentation
\raggedbottom  

% PACKAGES FOR TITLES
\usepackage{titlesec}
% \titlespacing{\section}{left spacing}{before spacing}{after spacing}
\titlespacing{\section}{0pt}{3.3ex}{2ex}
\titlespacing{\subsection}{0pt}{3.3ex}{1.65ex}
\titlespacing{\subsubsection}{0pt}{3.3ex}{1ex}
\usepackage{color}

% PACKAGES FOR LANGUAGE AND FONT
\usepackage[italian]{babel} % The document is in English  
\usepackage[utf8]{inputenc} % UTF8 encoding
\usepackage[T1]{fontenc} % Font encoding
\usepackage[11pt]{moresize} % Big fonts

% PACKAGES FOR IMAGES
\usepackage{graphicx}
\usepackage{transparent} % Enables transparent images
\usepackage{eso-pic} % For the background picture on the title page
\usepackage{subfig} % Numbered and caption subfigures using \subfloat.
\usepackage{tikz} % A package for high-quality hand-made figures.
\usetikzlibrary{}
\graphicspath{{./Images/}} % Directory of the images
\usepackage{caption} % Coloured captions
\usepackage{xcolor} % Coloured captions
\usepackage{amsthm,thmtools,xcolor} % Coloured "Theorem"
\usepackage{float}

% STANDARD MATH PACKAGES
\usepackage{amsmath}
\usepackage{amsthm}
\usepackage{amssymb}
\usepackage{amsfonts}
\usepackage{bm}
\usepackage[overload]{empheq} % For braced-style systems of equations.
\usepackage{fix-cm} % To override original LaTeX restrictions on sizes

% PACKAGES FOR TABLES
\usepackage{tabularx}
\usepackage{longtable} % Tables that can span several pages
\usepackage{colortbl}

% PACKAGES FOR ALGORITHMS (PSEUDO-CODE)
\usepackage{algorithm}
\usepackage{algorithmic}

% PACKAGES FOR REFERENCES & BIBLIOGRAPHY
\usepackage[colorlinks=true,linkcolor=black,anchorcolor=black,citecolor=black,filecolor=black,menucolor=black,runcolor=black,urlcolor=black]{hyperref} % Adds clickable links at references
\usepackage{cleveref}
\usepackage[square, numbers, sort&compress]{natbib} % Square brackets, citing references with numbers, citations sorted by appearance in the text and compressed
\bibliographystyle{abbrvnat} % You may use a different style adapted to your field

% OTHER PACKAGES
\usepackage{pdfpages} % To include a pdf file
\usepackage{afterpage}
\usepackage{lipsum} % DUMMY PACKAGE
\usepackage{fancyhdr} % For the headers
\fancyhf{}

% Input of configuration file. Do not change config.tex file unless you really know what you are doing. 
\input{Configuration_Files/config}

%----------------------------------------------------------------------------
%	NEW COMMANDS DEFINED
%----------------------------------------------------------------------------

% EXAMPLES OF NEW COMMANDS
\newcommand{\bea}{\begin{eqnarray}} % Shortcut for equation arrays
\newcommand{\eea}{\end{eqnarray}}
\newcommand{\e}[1]{\times 10^{#1}}  % Powers of 10 notation

%----------------------------------------------------------------------------
%	ADD YOUR PACKAGES (be careful of package interaction)
%----------------------------------------------------------------------------
\usepackage{tikz}
\usepackage{circuitikz}
\usepackage{subcaption}

%----------------------------------------------------------------------------
%	ADD YOUR DEFINITIONS AND COMMANDS (be careful of existing commands)
%----------------------------------------------------------------------------

%----------------------------------------------------------------------------
%	BEGIN OF YOUR DOCUMENT
%----------------------------------------------------------------------------

\begin{document}

\fancypagestyle{plain}{%
\fancyhf{} % Clear all header and footer fields
\fancyhead[RO,RE]{\thepage} %RO=right odd, RE=right even
\renewcommand{\headrulewidth}{0pt}
\renewcommand{\footrulewidth}{0pt}}

%----------------------------------------------------------------------------
%	TITLE PAGE
%----------------------------------------------------------------------------

\pagestyle{empty} % No page numbers
\frontmatter % Use roman page numbering style (i, ii, iii, iv...) for the preamble pages

\puttitle{
	title=Report 1: Banco di prova , % Title of the thesis
	course=086510 Progettazione Assistita dal Calcolatore,
	academicyear={2025-26}, 
}

%----------------------------------------------------------------------------
%	PREAMBLE PAGES: ABSTRACT (inglese e italiano), EXECUTIVE SUMMARY
%----------------------------------------------------------------------------
\startpreamble
\setcounter{page}{1} % Set page counter to 1

% TABLE OF CONTENTS
\thispagestyle{empty}
\tableofcontents % Table of contents 
\thispagestyle{empty}
\cleardoublepage

%-------------------------------------------------------------------------
%	THESIS MAIN TEXT
%-------------------------------------------------------------------------
% In the main text of your thesis you can write the chapters in two different ways:
%
%(1) As presented in this template you can write:
%    \chapter{Title of the chapter}
%    *body of the chapter*
%
%(2) You can write your chapter in a separated .tex file and then include it in the main file with the following command:
%    \chapter{Title of the chapter}
%    \input{chapter_file.tex}
%
% Especially for long thesis, we recommend you the second option.

\addtocontents{toc}{\vspace{2em}} % Add a gap in the Contents, for aesthetics
\mainmatter % Begin numeric (1,2,3...) page numbering

% --------------------------------------------------------------------------
% NUMBERED CHAPTERS % Regular chapters following
% --------------------------------------------------------------------------
\chapter{Problema reale}
\label{ch:problema_reale}%
\begin{figure}[H]
    \centering

    \begin{minipage}{0.3\linewidth}
        \centering
        \includegraphics[width=\linewidth]{Images/carr.png}
        \caption{Carrello a terra}
        \label{fig:carr}
    \end{minipage}
    \hfill
    \begin{minipage}{0.3\linewidth}
        \centering
        \includegraphics[width=\linewidth]{Images/Banchetto prova.png}
        \caption{Banco di prova con sistema di carico}
        \label{fig:banco}
    \end{minipage}
    \hfill
    \begin{minipage}{0.3\linewidth}
        \centering
        \includegraphics[width=\linewidth]{Images/cern.png}
        \caption{Cerniera a terra}
        \label{fig:cern}
    \end{minipage}

\end{figure}

Nel seguente rapporto, vogliamo studiare le deformazioni di una struttura a travi di un banco di prova, rappresentata nella figura \ref{fig:banco}.
Modelliamo le aste con elementi truss lineari (T3D2) e elementi beam lineari (B21) utilizzando il risolutore di simulazioni Abaqus e adottando un approccio analitico (solo per i truss).

Le 7 aste sono collegate con delle piastre bullonate, di cui le due di sinistra sono vincolate a terra con una cerniera (per quella inferiore) e un carrello (per quella superiore). Un carico di $428 \space N$ viene applicato dal sistema di carico a destra.

\section{Ottenimento dei vincoli}
\label{ss:ottenimento_vincoli}%
Le aste sono vincolate con delle piastre di giunzione bullonate, come in figura \ref{fig:giunz}.
Queste possono agire da vincolo di incastro quando ai bulloni viene applicata una coppia di serraggio di $10 \space Nm$, oppure da vincolo di cerniera se non sono serrati.

Otteniamo i vincoli di carrello e cerniera a terra con due montaggi a terra diversi, rispettivamente mostrati in figura \ref{fig:carr} e figura \ref{fig:cern}


\section{Travi e giunzioni}
\label{ss:travi_giunzioni}%
\begin{figure}[H]
    \centering

    \begin{minipage}{0.40\linewidth}
        \centering
        \includegraphics[width=0.6\linewidth]{Images/giunzione.png}
        \caption{Piastra di giunzione in attivo}
        \label{fig:giunz}
    \end{minipage}
    \hfill
    \begin{minipage}{0.40\linewidth}
        \centering
        \includegraphics[width=0.6\linewidth]{Images/Piastra tavola.png}
        \caption{Quote della piastra di giunzione}
        \label{fig:giunz_tavola}
    \end{minipage}

\end{figure}

Le varie dimensioni della piastra sono rappresentate nella figura \ref{fig:giunz_tavola}. I bulloni usati sono M6.
Nella figura \ref{fig:asta_giunta} abbiamo la rappresentazione del sistema di fissaggio delle aste.
\begin{figure}[H]
    \centering
    \includegraphics[width=0.50\linewidth]{Images/sist fissaggio.png}
    \caption{Sistema di fissaggio delle aste}
    \label{fig:asta_giunta}
\end{figure}

Le travi sono in alluminio, con il modulo di Young $E = 61639.8 MPa$, sono di sezione quadrata cava, con le quote riportate in figura \ref{fig:aste_quote}


\begin{figure}[H]
    \centering
    \includegraphics[width=1\linewidth]{Images/asta_quote.png}
    \caption{Quote di asta e sezione trasversale}
    \label{fig:aste_quote}
\end{figure}

\section{Posizionamento e utilizzo degli estensimetri}
\label{s:estensimetri}%
Posizioniamo degli estensimetri sulle aste della struttura, collegati a diversi canali, numerati da 0 a 7, connessi poi a una scheda di acquisizione \textit{National Instrument NI-9235} con software \textit{LabVIEW}.
Ogni estensimetro è di tipo \textit{1-LY13-6/120} , specifico per leghe di alluminio, montati come nell'immagine \ref{fig:est_montaggio}. Il circuito di misura è rappresentato nella figura \ref{fig:schema_wheatstone}

\begin{figure}[H]
    \centering
    \begin{tikzpicture}[scale=0.9]
\draw (-3,-2) to[battery] (-3, 2);

\draw (-3, 2) -- (0, 2); %to top
\draw (-3, -2) -- (0, -2);%to bot

\draw (0, 2) -- (0, 1.5);
\draw (0, -2) -- (0, -1.5);

\draw (0, 1.5) to[R=$R_1$] (-2, 0);
\draw (0, -1.5) to[R=$R_3$] (-2, 0);

\draw (0, 1.5) to[R=$R_2$] (2, 0);

\draw (0, -1.5) -- (3,-1.5);
\draw (0,0) node[draw, circle, minimum size=0.5cm] (v) {$V$};
\draw (-2, 0) -- (-0.4,0);
\draw (0.4, 0) -- (1,0);
\draw (1, 0) -- (1, -0.8);

\draw (2,0) -- (3,0);
\draw (3, 0) to[R=$R_{w,1}$] (4,0);
\draw (1, -0.8) -- (3, -0.8);
\draw (3, -0.8) to[R=$R_{w,2}$] (4,-0.8);

\draw (4, -0.8) -- (4.2, -0.8);
\draw (4, 0) -- (4.2, 0);
\draw (4.2, 0) -- (4.2, -0.8);

\draw (3, -1.5) to[R=$R_{w,3}$] (4, -1.5);
\draw (4, -1.5) -- (5.2, -1.5);

\draw (4.2, -0.4) -- (5.2, -0.4);

\draw (5.2, -0.4) -- (5.4, -0.4);
\draw (5.2, -1.5) -- (5.4, -1.5);

\fill (5.2,-0.4) circle (3pt);
\fill (5.2,-1.5) circle (3pt);

\draw (5.4, -0.4) -- (7, -0.4);
\draw[ultra thick] (7, -0.4) -- (7, -0.6);
\draw (7, -0.6) -- (5.4, -0.6);
\draw[ultra thick] (5.4, -0.6) -- (5.4, -0.8);
\draw (5.4, -0.8) -- (7, -0.8);
\draw[ultra thick] (7, -0.8) -- (7, -1);
\draw (7, -1) -- (5.4, -1);
\draw[ultra thick] (5.4, -1) -- (5.4, -1.3);
\draw (5.4, -1.3) -- (7, -1.3);
\draw[ultra thick] (7, -1.3) -- (7, -1.5);
\draw (7, -1.5) -- (5.4, -1.5);

%boxes
\draw[color=red] (-3.5, 3) rectangle (2.7, 0-3);
\draw [ color=red](-0.75,3.3) to[short] (-0.75,3.3) node {Scheda acquisizione};

\draw[color=blue] (2.9, 1.2) rectangle (4.6, -2.5);
\draw [ color=blue](3.75, 1.4) to[short] (3.75, 1.4) node {Fili};

\draw[color=purple] (4.9, 0.1) rectangle (7.3, -2);
\draw [color=purple](6.1, 0.3) to[short] (6.1, 0.4) node {Estensimetro};

\end{tikzpicture}
\caption{Schema del ponte di Wheatstone con estensimetro a 3 fili}
\label{fig:schema_wheatstone}
\end{figure}

I nostri estensimetri sono posizionati, uno per asta, come nell'immagine \ref{fig:posiz_est}


\begin{figure}[H]
    \centering

    \begin{minipage}{0.45\linewidth}
        \centering
        \includegraphics[width=\linewidth]{Images/Montaggio estensimetro.png}
        \caption{Montaggio estensimetro}
        \label{fig:est_montaggio}
    \end{minipage}
    \hfill
    \begin{minipage}{0.45\linewidth}
        \centering
        \includegraphics[width=0.65\linewidth]{Images/posizioni_est.png}
        \caption{Posizioni degli estensimetri}
        \label{fig:posiz_est}
    \end{minipage}

\end{figure}

I dettagli della configurazione corrente sono:
\begin{itemize}
    \item Gli estensimetri misurano uno alla volta
    \item Gli effetti termici sull'estensimetro non sono compensati
    \item Le resistenze dei fili non sono compensate
\end{itemize}


\chapter{Modellazione con elementi truss}
\label{ch:mod_truss}%
Studiamo la struttura con elementi lineari truss: due nodi e forze solo assiali

\section{Trattazione analitica}
\label{s:truss_analog}%
\begin{figure}[H]
    \centering
    \includegraphics[width=0.6\linewidth]{Images/schema amalitico.png}
    \caption{Esploso analitico con elementi truss, forze puramente assiali}
    \label{fig:esploso_an}
\end{figure}

Usiamo una formulazione analitica, quindi dato lo schema delle forze nell'esploso in figura \ref{fig:esploso_an} possiamo ricavarci la matrice in figura \ref{fig:matr} del bilancio delle forze ai nodi, dalla quale ricaviamo le nostre forze interne e forze vincolari:

$$F_1 = -428\space N \quad F_2 = 428 \space N\quad F_3 = -428 \space N \quad F_4 = 428 \space N \quad F_5 = 428 \space N$$
$$F_6 = -856 \space N\quad F_7 = 0 \space N\quad R_{1x} = -428\sqrt3\space N \quad R_{1y} = -428 \space N\quad R_{2x} = -428\sqrt3 \space N$$

\begin{figure}[H]
    \centering
    \begin{gather*}
    \left\{
        \begin{array}{@{}c r r r r r r r r r r @{=} c}
P & & & & & & & & & & &428\\
P &+\frac{1}2 F_1 & -\frac12F_2 & & & & & & & & & 0\\
& +\frac{\sqrt3}{2}F_1 & +\frac{\sqrt 3}{2}F_2 & & & & & & & & & 0\\
& & +\frac{1}{2}F_2 & +F_3 & + \frac{1}{2}F_4& & & & & & & 0\\
& & +\frac{\sqrt{3}}{2}F_2 & & -\frac{\sqrt{3}}{2}F_4& & & & & & & 0\\
& +\frac{1}{2}F_1 & & +F_3 & & +\frac{1}{2}F_5 & -\frac{1}{2}F_6& & & & & 0\\
& \frac{\sqrt{3}}{2}F_1 & & & &-\frac{\sqrt{3}}{2}F_5 & -\frac{\sqrt{3}}{2}F_6& & & & & 0\\
& & & & & & \frac{\sqrt{3}}{2}F_6 & & +R_{1x}& & & 0\\
& & & & \frac{\sqrt{3}}{2}F_4 & + \frac{\sqrt{3}}{2}F_5 & & & & & +R_{2x} & 0\\
& & & & \frac{1}{2} F_4 & -\frac{1}{2}F_5 & & -F_7 & & & & 0\\
& & & & & & +\frac{1}{2}F_6 & +F_7 & & -R_{1y} & & 0
\end{array}
    \right.
    \end{gather*}
    \caption{Matrice del bilancio delle forze per il caso con forze solo assiali}
    \label{fig:matr}
\end{figure}



\section{Simulazione ad elementi finiti}
\label{s:truss_sim}%

\chapter{Modellazione con elementi beam}
\label{ch:mod_beam}%

\section{Travi incastrate}
\label{s:full_encastre}%

\section{Travi incernierate}
\label{s:beam_2}%

\chapter{Risultato sperimentale}
\label{ch:exp}%

\end{document}
